%% LyX 2.2.4 created this file.  For more info, see http://www.lyx.org/.
%% Do not edit unless you really know what you are doing.
\documentclass[oneside,english]{book}
\usepackage[T1]{fontenc}
\setcounter{secnumdepth}{3}
\setcounter{tocdepth}{3}
\usepackage{amsmath}
\usepackage{amsthm}

\makeatletter
%%%%%%%%%%%%%%%%%%%%%%%%%%%%%% Textclass specific LaTeX commands.
\theoremstyle{plain}
\ifx\thechapter\undefined
\newtheorem{thm}{\protect\theoremname}
\else
\newtheorem{thm}{\protect\theoremname}[chapter]
\fi
  \theoremstyle{definition}
  \newtheorem{problem}[thm]{\protect\problemname}

%%%%%%%%%%%%%%%%%%%%%%%%%%%%%% User specified LaTeX commands.
\usepackage{pgfplots}
\usepackage{enumitem}
\setenumerate{label=A)}
\setenumerate[2]{label=a)}
% Added by lyx2lyx
\renewcommand{\textendash}{--}
\renewcommand{\textemdash}{---}

\makeatother

\usepackage{babel}
  \providecommand{\problemname}{Problem}
\providecommand{\theoremname}{Theorem}

\begin{document}

\part{Probabilidad}

\chapter{Definiciones b\'{a}sicas}
\begin{problem}
Sean $A,B,C$ eventos aleatorios. Identifique las siguientes ecuaciones
y frases
\end{problem}

\begin{problem}
A partir de los axiomas, pruebe la propuedad \textbf{P5}: 
\[
P\left(\bigcup_{n=1}^{\infty}A_{n}\right)\leq\sum_{n=1}^{\infty}P\left(A_{n}\right)
\]
Consideremos $B_{k}=\bigcup_{n=k}^{\infty}A_{n}$ entonces 
\begin{align*}
P\left(\bigcup_{n=1}^{\infty}A_{n}\right) & =P\left(\bigcup_{n=1}^{k}A_{n}\cup B_{k+1}\right)\\
 & \leq P\left(\bigcup_{n=1}^{k}A_{n}\right)+P\left(B_{k+1}\right)\\
 & \leq\sum_{n=1}^{k}P\left(A_{n}\right)+P\left(B_{k+1}\right)\\
 & =\lim_{k\to\infty}\sum_{n=1}^{k}P\left(A_{n}\right)+P\left(B_{k+1}\right)\\
 & =\sum_{n=1}^{\infty}P\left(A_{n}\right)+\lim_{k\to\infty}P\left(B_{k+1}\right)\\
 & \leq\sum_{n=1}^{\infty}P\left(A_{n}\right)
\end{align*}
\end{problem}

\begin{problem}
Sean $A_{1},A_{2},\ldots$ eventos aleatorios, mostrar que: 
\begin{enumerate}
\item $P\left(\bigcap_{k=1}^{n}A_{k}\right)\geq1-\sum_{k=1}^{n}P\left(A_{k}^{c}\right)$
\item Si $P\left(A_{k}\right)\geq1-\varepsilon\implies P\left(\bigcap_{k=1}^{n}A_{k}\right)\geq1-n\varepsilon$
\item $P\left(\bigcap_{k=1}^{\infty}A_{k}\right)\geq1-\sum_{k=1}^{\infty}P\left(A_{k}^{c}\right)$
\end{enumerate}
\end{problem}


\part{Inferencia Cl\'{a}sica}

%% LyX 2.2.4 created this file.  For more info, see http://www.lyx.org/.
%% Do not edit unless you really know what you are doing.
\documentclass[oneside,english]{book}
\usepackage[T1]{fontenc}
\setcounter{secnumdepth}{3}
\setcounter{tocdepth}{3}
\usepackage{amsmath}
\usepackage{amsthm}

\makeatletter
%%%%%%%%%%%%%%%%%%%%%%%%%%%%%% Textclass specific LaTeX commands.
\theoremstyle{plain}
\ifx\thechapter\undefined
\newtheorem{thm}{\protect\theoremname}
\else
\newtheorem{thm}{\protect\theoremname}[chapter]
\fi
  \theoremstyle{definition}
  \newtheorem{problem}[thm]{\protect\problemname}

%%%%%%%%%%%%%%%%%%%%%%%%%%%%%% User specified LaTeX commands.
\usepackage{pgfplots}
\usepackage{enumitem}
\setenumerate{label=A)}
\setenumerate[2]{label=a)}
% Added by lyx2lyx
\renewcommand{\textendash}{--}
\renewcommand{\textemdash}{---}

\makeatother

\usepackage{babel}
  \providecommand{\problemname}{Problem}
\providecommand{\theoremname}{Theorem}

\begin{document}

\chapter{Reducci\'{o}n de datos y estad\'{\i}sticos suficientes}
\begin{problem}
Ejemplo de estad\'{\i}sticos suficientes mediante teorema de factorizaci\'{o}n
\end{problem}


\end{document}

\end{document}
